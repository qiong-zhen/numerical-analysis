\documentclass[a4paper]{article}
\usepackage{ctex} 
\usepackage[affil-it]{authblk}
\usepackage[backend=bibtex,style=numeric]{biblatex}
\usepackage{amsmath}

\usepackage{geometry}
\geometry{margin=1.5cm, vmargin={0pt,1cm}}
\setlength{\topmargin}{-1cm}
\setlength{\paperheight}{29.7cm}
\setlength{\textheight}{25.3cm}

\addbibresource{citation.bib}

\begin{document}
% =================================================
\title{Numerical Analysis homework 2}

\author{谭希 Tan Xi 3220100027
  \thanks{Electronic address: \texttt{3220100027@zju.edu.cn}}}
\affil{(数学与应用数学2201), Zhejiang University }


\date{Due time: \today}

\maketitle

% ============================================
\section*{I. }
\subsection*{I-a}
The first derivative is:
\[f'(x) = -\frac{1}{x^2}\]
The second derivative is:
\[f''(x) = \frac{2}{x^3}\]
Using the interpolation error formula:
\[f(x) - p_1(f; x) = \frac{f''(\xi(x))}{2} (x - 1)(x - 2)\]
then:
\[\xi(x)^3 = \frac{(x - 1)(x - 2)}{f(x) - p_1(f; x)}\]

\[\xi(x)^3 = \frac{1}{\frac{1}{x} - \left(\frac{3 - x}{2}\right)}(x - 1)(x - 2).\]

\[\xi(x)^3 = \frac{1}{\frac{(x - 1)(x - 2)}{2x}}(x - 1)(x - 2).\]

Thus:
\[\xi(x)^3 = 2x,\]

\[\xi(x) = \sqrt[3]{2x}.\]

\subsection*{I-b}
According to I-a,$\xi(x)=\sqrt[3]{2x}$ ,increases .$f''(\xi(x))=\frac{1}{x}$, decreases.\\
So $\max\xi(x)=\sqrt[3]{4}$ and $\min\xi(x)=\sqrt[3]{2}$.
$\max f''(\xi(x))=1$

\section*{II. }

1. Lagrange Interpolation Polynomial:\\
   The Lagrange interpolation polynomial \( p(x) \) that fits the points \( (x_i, f_i) \) is given by:
   \[
   p(x) = \sum_{i=0}^{n} f_i L_i(x)
   \]
   where \( L_i(x) \) is the Lagrange basis polynomial defined as:
   \[
   L_i(x) = \prod_{\substack{0 \leq j \leq n \\ j \neq i}} \frac{x - x_j}{x_i - x_j}
   \]
   Each \( L_i(x) \) is a polynomial of degree \( n \), thus \( p(x) \) is a polynomial of degree at most \( n \).

2. Ensuring Non-negativity:\\
   To ensure that the interpolation polynomial \( p(x) \) is non-negative for all \( x \in \mathbb{R} \), we can utilize the squares of the Lagrange basis polynomials:
   \[
   q(x) = \sum_{i=0}^{n} f_i (L_i(x))^2
   \]
   \( (L_i(x))^2 \) ensures that each term is non-negative, as squares of real numbers are non-negative. The polynomial \( q(x) \) will be of degree \( 2n \) because the degree of \( (L_i(x))^2 \) is \( 2n \).

   Therefore, we can express  polynomial \( p(x) \) in \( P^+_{2n} \) as:
   \[
   p(x) = \sum_{i=0}^{n} f_i (L_i(x))^2
   \]
   
\[
p(x) = \sum_{i=0}^{n} f_i \left( \prod_{\substack{0 \leq j \leq n \\ j \neq i}} \frac{x - x_j}{x_i - x_j} \right)^2
\]
This polynomial is in \( P^+_{2n} \) and satisfies the required conditions.


\section*{III. }
\subsection*{III-a}

For \( n = 0 \):
\[f[t] = e^t\]
The right side becomes:
\[\frac{(e - 1)^0}{0!} e^t = 1 \cdot e^t = e^t\]
Thus, the base case holds.

Assume the statement is true for \( n = k \):
\[f[t, t + 1, \ldots, t + k] = \frac{(e - 1)^k}{k!} e^t\]
We need to prove it for \( n = k + 1 \):
\[f[t, t + 1, \ldots, t + (k + 1)] = f[t, t + 1, \ldots, t + k] + f[t + k + 1]\]

\[f[t, t + 1, \ldots, t + k] = \frac{(e - 1)^k}{k!} e^t\]
and 
\[f[t + k + 1] = e^{t + k + 1} = e^t e^{k + 1} = e^{t} e^{k + 1}\]

Then:
\[f[t, t + 1, \ldots, t + (k + 1)] = \frac{(e - 1)^k}{k!} e^t + e^{t} e^{k + 1}\]

\[f[t, t + 1, \ldots, t + (k + 1)] = e^t \left( \frac{(e - 1)^k}{k!} + e^{k + 1} \right)\]

Notice that:
\[e^{k + 1} = \frac{(e - 1)^{k + 1}}{(k + 1)!}\]
\[e^{k + 1} = \frac{(e - 1)^{k + 1}}{(k + 1)!} = \frac{(e - 1)^k (e - 1)}{(k + 1)!}\]
Thus, the equation becomes:
\[f[t, t + 1, \ldots, t + (k + 1)] = e^t \left( \frac{(e - 1)^k}{k!} + \frac{(e - 1)^{k + 1}}{(k + 1)!} \right)\]

\[f[t, t + 1, \ldots, t + (k + 1)] = e^t \left( \frac{(e - 1)^k (k + 1) + (e - 1)^{k + 1}}{(k + 1)!} \right)\]
\[= e^t \frac{(e - 1)^k (k + 1) + (e - 1)^{k + 1}}{(k + 1)!} = e^t \frac{(e - 1)^{k + 1} + (e - 1)^{k + 1}}{(k + 1)!} = e^t \frac{(e - 1)^{k + 1}}{(k + 1)!}\]
Thus:
\[f[t, t + 1, \ldots, t + (k + 1)] = \frac{(e - 1)^{k + 1}}{(k + 1)!} e^t\]

By mathematical induction, we have shown that:
\[f[t, t + 1, \ldots, t + n] = \frac{(e - 1)^n}{n!} e^t \quad \forall n \in \mathbb{N}\]

\subsection*{III-b}
\[f[0, 1, \ldots, n] = \frac{1}{n!} f^{(n)}(\xi)\]
Thus,
\[f[0, 1, \ldots, n] = \frac{(e - 1)^n}{n!}\]
\[\frac{(e - 1)^n}{n!} = \frac{1}{n!} f^{(n)}(\xi)\]
\[f^{(n)}(\xi) = (e - 1)^n\]
Then:
\[f^{(n)}(x) = e^x\]
\[f^{(n)}(\xi) = e^\xi\]
\[e^\xi = (e - 1)^n\]
\[\xi = \ln((e - 1)^n) = n \ln(e - 1)\]

Since \( \ln(e - 1) \) is a constant, we need to compare \( n \ln(e - 1) \) with \( \frac{n}{2} \)

\[e - 1 \approx 1.71828 \quad \Rightarrow \quad \ln(e - 1) \approx 0.54132\]
Since \( 0.54132 > 0.5 \):
\[n \ln(e - 1) > \frac{n}{2}\]

Thus, \( \xi \) is located to the right of the midpoint \( \frac{n}{2} \).

\section*{IV. }
\subsection*{IV-a}
The table of divided differences:
$$
\begin{tabular}{c|ccccc}
    $x_0 = 0$ & 1 & & & & \\
    $x_1 = 1$ & 2 & 1 & & & \\
    $x_2 = 1$ & 2 & -1 & -2 & & \\
    $x_3 = 3$ & 0 & -1 & 0 & 0.6667 & \\
    $x_4 = 3$ & 0 & 0 & 0.5 & 0.25 & -0.1389 \\
\end{tabular}
$$
Then:$p_3(f;x)=5-2(x-0)+1(x-0)(x-1)+0.25(x-0)(x-1)(x-3)=\frac{1}{4}x^3-\frac{9}{4}x+5$.

\subsection*{IV-b}
 $$
   \begin{aligned}
    p_3'(f;x) &= -2 + x - 1 + x + 0.25 \left[ (x - 1)(x - 3) + x(x - 3) + x(x - 1) \right]   \\
    &= 0.75x^2 - 2.25.
   \end{aligned}
   $$

   Let $p_3'(f;x) = 0$,
   \[x = \sqrt{\frac{2.25}{0.75}} = \sqrt{3}\]

   Therefore, the approximate location of the minimum $x_{\text{min}}$ is around $\sqrt{3}$.

\section*{V. }
\subsection*{V-a}
The table of divided differences:
\[
\begin{array}{c|cccccc}
  0 & 0 & & & & & \\
  1 & 1 & 1 & & & & \\
  1 & 1 & 7 & 6 & & & \\
  1 & 1 & 7 & 21 & 15 & & \\
  2 & 128 & 127 & 120 & 99 & 42 & \\
  2 & 128 & 448 & 321 & 201 & 102 & 30 \\
\end{array}
\]

Thus, \(f[0,1,1,1,2,2]=30\).

\subsection*{V-b}
By \textbf{Corollary 2.22}, $\exists \xi \in (0,2)\ s.t.\frac{f^{(5)}(\xi)}{5!}=f[0,1,1,1,2,2]=3600$.

Thus:$f^{(5)}(x)=7·6·5·4·3x^2=2520x^2 \\
\Rightarrow \xi =\sqrt{\frac{10}{7}} \approx 1.195$

\section*{VI. }
\subsection*{VI-a}
The table of divided differ:
\[
\begin{array}{c|ccccc}
  0 & 1 & & & & \\
  1 & 2 & 1 & & & \\
  1 & 2 & -1 & -2 & & \\
  3 & 0 & -1 & 0 & \frac{2}{3} & \\
  3 & 0 & 0 & \frac{1}{2} & \frac{1}{4} & -\frac{5}{36} \\
\end{array}
\]

Then:$p(x)=1+1x-2x(x-1)+\frac{2}{3}x(x-1)^2-\frac{5}{36}x(x-1)^2(x-3)$.

$f(2) \approx p(2)=1+2-4+\frac{4}{3}+\frac{5}{18}=\frac{11}{18}$.

\subsection*{VI-b}
$$
f(x) - p_N(f; x) = \frac{f^{(N+1)(\xi)}}{(N+1)!} \prod_{i=0}^k(x-x_i)^{m_i + 1},
$$

for some $\xi \in (a,b)$.$|f^{(5)}(\xi)| \le M$ on $[0, 3]$. The error of the above question:

$$
|f(2) - p(2)| \le \left| \frac{M}{5!} (2 - 0)(2 - 1)^2(2-3)^2 \right| = \frac{M}{60}.
$$

\section*{VII. }

\subsection*{a. Proof for Forward Difference}

We aim to prove that:
\[\Delta^k f(x) = k! h^k f[x_0, x_1, \dots, x_k]\]
The forward difference operator  \( \Delta f(x) \)is defined as:
\[\Delta f(x) = f(x + h) - f(x)\]

The second forward difference is:
\[\Delta^2 f(x) = \Delta(\Delta f(x)) = \Delta f(x + h) - \Delta f(x) = [f(x + 2h) - f(x + h)] - [f(x + h) - f(x)]\]
\[\Delta^2 f(x) = f(x + 2h) - 2f(x + h) + f(x)\]

The k-th forward difference is:

\[\Delta^k f(x) = \sum_{j=0}^k (-1)^{k-j} \binom{k}{j} f(x + jh)\]

The divided difference \( f[x_0, x_1, \dots, x_k] \)  is a recursive formula defined as:

\[f[x_j] = f(x_j)\]
\[f[x_j, x_{j+1}, \dots, x_{j+k}] = \frac{f[x_{j+1},\dots, x_{j+k}] - f[x_j, \dots, x_{j+k-1}]}{x_{j+k} - x_j}\]

If \( x_j = x + jh \) ,then\( x_{j+k} - x_j = kh \)

The k-th forward difference is related to the k-th divided difference as follows:

\[\Delta^k f(x) = \sum_{j=0}^k (-1)^{k-j} \binom{k}{j} f(x + jh)\]

By expanding the divided differences in terms of forward differences:
\[\Delta^k f(x) = k! h^k f[x_0, x_1, \dots, x_k]\]

\subsection*{b. Proof for Backward Difference}

\[\nabla f(x) = f(x) - f(x - h)\]

The second backward difference is:
\[\nabla^2 f(x) = \nabla(\nabla f(x)) = \nabla f(x) - \nabla f(x - h) = [f(x) - f(x - h)] - [f(x - h) - f(x - 2h)]\]
\[\nabla^2 f(x) = f(x) - 2f(x - h) + f(x - 2h)\]

the  \( k \) -th backward difference is:

\[\nabla^k f(x) = \sum_{j=0}^k (-1)^j \binom{k}{j} f(x - jh)\]

Expand the divided differences for the points \( x_0, x_{-1}, \dots, x_{-k} \) 。The \( k \)-th backward difference can then be expressed in terms of the\( k \)-th divided difference as:

\[\nabla^k f(x) = k! h^k f[x_0, x_{-1}, \dots, x_{-k}]\]

\section*{VIII. }
The divided difference \(f[x_0, x_1, \dots, x_n]\) is recursively defined as:

\[f[x_0] = f(x_0),\]
\[f[x_0, x_1, \dots, x_n] = \frac{f[x_1, \dots, x_n] -f[x_0, \dots, x_{n-1}]}{x_n - x_0}.\]
Then:
\[\frac{\partial}{\partial x_0} f[x_0, x_1, \dots, x_n] = \frac{\partial}{\partial x_0} \left( \frac{f[x_1, \dots,x_n] - f[x_0, \dots, x_{n-1}]}{x_n - x_0} \right).\]

\[\frac{\partial}{\partial x_0} \left( \frac{f[x_1, \dots, x_n] - f[x_0, \dots, x_{n-1}]}{x_n - x_0} \right)
= \frac{(x_n - x_0) \frac{\partial}{\partial x_0} \left( f[x_1, \dots, x_n] - f[x_0, \dots, x_{n-1}] \right) + f[x_1, \dots, x_n] - f[x_0, \dots, x_{n-1}]}{(x_n - x_0)^2}.\]

Since \(f[x_1, \dots, x_n]\) does not depend on \(x_0\), its derivative is zero.
\[\frac{\partial}{\partial x_0} f[x_0, x_1, \dots, x_n] = \frac{f[x_0, \dots, x_{n-1}]}{x_n - x_0}.\]

\[\frac{\partial}{\partial x_0} f[x_0, x_1, \dots, x_n] = f[x_0, x_0, x_1, \dots, x_n].\]
Thus:
\[\frac{\partial}{\partial x_i} f[x_0, x_1, \dots, x_n] = f[x_0, \dots, x_i, x_i, \dots, x_n].\]

\section*{IX. }
\( \forall p \in \tilde{P}_n \)
\[\max_{x \in [-1,1]} \left| \frac{T_n(x)}{2^{n-1}} \right| \leq \max_{x \in [-1,1]} |p(x)|\]
 \( T_n(x) \) denotes the \( n \)-th Chebyshev polynomial.

\[\min \max_{y \in [-1,1]} |y^n+b_1y^{n-1}+\ldots+b_n| \geq \frac{1}{2^{n-1}}\]

For $x \in [a,b]$ , take $x=\frac{b-a}{2}y+\frac{b+a}{2},\ y \in [-1,1]$. Then,

\[\min \max_{x \in [a,b]} \left|a_0 x^n + a_1 x^{n-1} + \ldots + a_n \right|
= \min \max_{y \in [-1,1]} \frac{|a_0| (b-a)^n}{2^n} \left| y^n + b_1 y^{n-1} + \ldots + b_n \right|\]

\[\min \max_{x \in [a, b]} \left| a_0 x^n + a_1 x^{n-1} + \cdots + a_n \right|
= \frac{|a_0| (b-a)^n}{2^{2n-1}}\]

\section*{X. }
\[T_n(x) = \cos(n \arccos(x))\]

Recall that \(T_n(x)\) has the property:
    \[|T_n(x)| \leq 1 \quad \text{for } x \in [-1, 1].\]
    Therefore, we also have:
    \[|T_n(a)| \geq 1 \quad \text{for } a > 1.\]
    Hence, we can state:
    \[|\hat{p}_n(x)| = \left|\frac{T_n(x)}{T_n(a)}\right| \leq \frac{|T_n(x)|}{|T_n(a)|} \leq \frac{1}{|T_n(a)|} \quad \text{for } x \in [-1, 1].\]

 Let \(p \in P^a_n\). Then by definition:\[p(a) = 1.\]
    For \(x \in [-1, 1]\), we have the following estimation:
    \[\|p\|_\infty = \max_{x \in [-1, 1]} |p(x)| \geq |p(a)| = 1.\]

    \[ \|\hat{p}_n\|_\infty = \max_{x \in [-1, 1]} \left| \frac{T_n(x)}{T_n(a)} \right| \leq \frac{1}{|T_n(a)|}.\]
    Given that \(T_n(a) \geq 1\) for \(a > 1\), we conclude that:
    \[\|\hat{p}_n\|_\infty \leq \frac{1}{\|p\|_\infty} \|p\|_\infty = \|p\|_\infty.\]
    This shows that:
    \[\|\hat{p}_n\|_\infty \leq \|p\|_\infty.\]

\section*{XI. \[b_{n,k}(t):=\binom{n}{k} t^k (1-t)^{n-k},\ \ k = 0, 1, \ldots, n,\ \ t \in [0,1]\]
Proof: \[ b_{n-1,k}(t)= \frac{n-k}{n} b_{n,k}(t)+\frac{k+1}{n} b_{n,k+1}(t)\]. }

$$
\forall k = 0, 1, \dots n,\ b_{n, k}(t)=\binom{n}{k}t^k(1-t)^{n-k},
$$

Then, 
$$\begin{aligned}
   \frac{n-k}{n}b_{n,k}(t)+\frac{k+1}{n}b_{n,k+1}(t)
   =& \frac{n-k}{n} \frac{n!}{k!(n-k)!} t^k(1-t)^{n-k}+ \frac{k+1}{n} \frac{n!}{(k+1)!(n-k-1)!} t^{k+1}(1-t)^{n-k-1} \\
   =& \frac{(n-1)!}{k!(n-k-1)!} t^k(1-t)^{n-k} +\frac{(n-1)!}{k!(n-k-1)!} t^{k+1}(1-t)^{n-k-1} \\
   =& \binom{n-1}{k} t^k(1-t)^{n-k-1}(1 - t + t)   \\
   =& \binom{n-1}{k} t^k(1-t)^{n-k-1}  \\
   =& b_{n-1, k}(t).
\end{aligned}$$

\section*{XII.Proof: \[\int_{0}^{1} b_{n,k}(t) \, dt = \frac{1}{n+1}.\] }

We know that:\[ B(p, q) = \int_0^1 t^{p-1} (1-t)^{q-1}dt,\ p>0,q>0\]
Thus:
\begin{align*}
  \int_0^1 b_{n,k}(t) &= \binom{n}{k}\int_0^1 t^k (1-t)^{n-k}dt \\
  &= \binom{n}{k}B(k+1,n-k+1) \\
  &= \frac{n!}{(k+1)!(n-k)!} \left[ \left. t^{k+1}(1-t)^{n-k} \right|_{0}^{1} - (n-k)\int_{0}^{1} t^{k+1} (1-t)^{n-k-1} dt  \right]\\
   &= \frac{n!}{(k+1)!(n-k-1)!} \int_0^1t^{k+1}(1-t)^{n-k-1}dt    \\
   &= \int_0^1 t^n \, dt \\
   &= \left[ \frac{t^n}{n+1} \right]_0^1\\
  &= \frac{1}{n+1} \\
\end{align*}

% ===============================================

\end{document}