\documentclass[a4paper]{article}
\usepackage{ctex} 
\usepackage[affil-it]{authblk}
\usepackage[backend=bibtex,style=numeric]{biblatex}
\usepackage{amsmath}

\usepackage{geometry}
\geometry{margin=1.5cm, vmargin={0pt,1cm}}
\setlength{\topmargin}{-1cm}
\setlength{\paperheight}{29.7cm}
\setlength{\textheight}{25.3cm}

\addbibresource{citation.bib}

\begin{document}
% =================================================
\title{Numerical Analysis Report1}

\author{谭希 3220100027
  \thanks{Electronic address: \texttt{3220100027@zju.edu.cn}}}
\affil{(数学与应用数学2201), Zhejiang University }


\date{Due time: \today}

\maketitle

% ============================================
\section{A}
Defining a base class \texttt{EquationSolver} and three derived classes: \texttt{Bisection\_Method}, \texttt{Newton\_Method}, and \texttt{Secant\_Method}.\n

\subsection{Bisection\_Method}
Implements the Bisection Method for finding the roots of a function.

\begin{itemize}
    \item First, checks if the function values at the endpoints \texttt{a} and \texttt{b} have opposite signs to ensure the existence of a root.
    \item Continuously narrows down the interval \([a, b]\) to find the root until the width is less than \texttt{eps} or a sufficiently small function value is found.
    \item Throws an exception if the maximum iteration limit is reached.
\end{itemize}

\subsection{Newton\_Method}
Implements Newton's Method for finding the roots of a function.
\begin{itemize}
    \item Uses the Newton iteration formula \(x = x - \frac{f(x)}{f'(x)}\) to iterate.
    \item Stops iterating if the function value is less than or equal to \texttt{eps}.
    \item Computes the derivative; if the derivative is less than \texttt{eps}, throws an exception to avoid division by zero errors.
    \item Throws an exception if the maximum iteration limit is reached.
\end{itemize}

\subsection{Secant\_Method}
Implements the Secant Method for finding the roots of a function.
\begin{itemize}
    \item Uses the Secant method formula to update \(x_2\) until convergence is achieved.
    \item Checks if the function values at \texttt{x0} and \texttt{x1} are too close; if so, throws an exception.
    \item Throws an exception if the maximum iteration limit is reached.
\end{itemize}

\section{B}
\subsection{Function 1: \(f_1(x) = x^{-1} - \tan(x)\) on \([0, \frac{\pi}{2}]\)}
\begin{itemize}
    \item \textbf{Implementation:} The Bisection Method is used to find the root of this function, avoiding singularities at \(x = 0\) and \(x = \frac{\pi}{2}\).
    \item \textbf{Output:} A root is: 0.860334
\end{itemize}

\subsection{Function 2: \(f_2(x) = x^{-1} - 2^x\) on \([0, 1]\)}
\begin{itemize}
    \item \textbf{Implementation:} The Bisection Method is utilized, avoiding the singularity at \(x = 0\).
    \item \textbf{Output:}A root is: 0.641186
\end{itemize}

\subsection{Function 3: \(f_3(x) = 2^{-x} + e^x + 2\cos(x) - 6\) on \([1, 3]\)}
\begin{itemize}
    \item \textbf{Output:} A root is: 1.82938
\end{itemize}

\subsection{Function 4: \(f_4(x) = \frac{x^3 + 4x^2 + 3x + 5}{2x^3 - 9x^2 + 18x - 2}\) on \([0, 4]\)}
\begin{itemize}
    \item \textbf{Output:} A root is: 0.117877
\end{itemize}

\section{C}
Root Values: The actual root values depend on the implementation of\texttt{Newton\_Method::solve()} and how it converges to a solution based on the initial guess.\\
    Given the nature of the function \(x = \tan(x)\):
    \begin{itemize}
        \item The root close to \textbf{4.5} is likely around \textbf{4.493}.
        \item The root close to \textbf{7.7} is likely around \textbf{7.725}.
    \end{itemize}
    
\section{D}
\subsection{Secant Method Solution}

To use the Secant Method to solve the equations, we created a \texttt{Secant\_Method} class that takes a function instance and two initial values as inputs. The \texttt{solve()} method of this class implements the core logic of the Secant Method, iteratively calculating until the desired precision is reached or the maximum number of iterations is exceeded.

\subsection{Results}
Upon executing the program, we successfully found the roots of the three equations. The results for each equation are as follows:
\begin{itemize}
    \item \textbf{For \( f(x) = \sin\left(\frac{x}{2}\right) - 1 \)}:
    \begin{itemize}
        \item Found root: \( x \approx 3.14093 \)
    \end{itemize}
    
    \item \textbf{For \( f(x) = e^x - \tan(x) \)}:
    \begin{itemize}
        \item Found root: \( x \approx 1.30633 \)
    \end{itemize}
    
    \item \textbf{For \( f(x) = x^3 - 12x^2 + 3x + 1 \)}:
    \begin{itemize}
        \item Found root: \( x \approx -0.188685 \)
    \end{itemize}
\end{itemize}

\section{E}
The equation used to compute the volume of the segment is defined as follows:
\[
f(h) = L \left( 0.5 \pi r^2 - r^2 \arcsin\left(\frac{h}{r}\right) - h \sqrt{r^2 - h^2} \right) - V_{\text{target}}
\]
where \( h \) is the height of the segment being calculated.

\subsection{ Bisection Method}
This method divides the interval in which the root lies into two halves, continually narrowing down the search area.
\begin{itemize}
    \item Initial interval: \( [0.0, r] \) (i.e., [0.0, 1.0]).
    \item Tolerance levels: \( 1e-7 \) for function value and \( 1e-6 \) for the interval size.
\end{itemize}

\subsection{ Newton's Method}
This method uses the derivative of the function to iteratively find successively better approximations of the root.
\begin{itemize}
    \item Initial guess: \( h = 0.5 \).
\end{itemize}

\subsection{ Secant Method}
This method uses two initial guesses to approximate the derivative and iteratively find the root.
\begin{itemize}
    \item Initial guesses: \( h_0 = 0.1 \) and \( h_1 = 0.9 \).
\end{itemize}

\textbf{Results:}
\begin{itemize}
    \item \textbf{Bisection Method}: Result: \( h \approx 0.166166 \, \text{ft} \)
    \item \textbf{Newton's Method}: Result: \( h \approx 0.166166 \, \text{ft} \)
    \item \textbf{Secant Method}: Result: \( h \approx 0.166166 \, \text{ft} \)
\end{itemize}

\section{F}
The equation \( f(\alpha) \) is defined as follows:
\begin{equation}
    f(\alpha) = A \sin(\alpha) \cos(\alpha) + B \sin^2(\alpha) - C \cos(\alpha) - E \sin(\alpha)
\end{equation}
where:
\begin{align*}
    A &= l \cdot \sin(\beta_1) \\
    B &= l \cdot \cos(\beta_1) \\
    C &= (h + 0.5 \cdot D) \cdot \sin(\beta_1) - 0.5 \cdot D \cdot \tan(\beta_1) \\
    E &= (h + 0.5 \cdot D) \cdot \cos(\beta_1) - 0.5 \cdot D
\end{align*}

\subsection{Newton's Method Results}
\begin{itemize}
    \item \textbf{Using depth \( D_1 = 55 \) inches and initial guess \( 33^\circ \)}:
    \begin{itemize}
        \item Solution: \( \alpha \approx 32.9722^\circ \)
    \end{itemize}
    
    \item \textbf{Using depth \( D_2 = 30 \) inches and initial guess \( 33^\circ \)}:
    \begin{itemize}
        \item Solution: \( \alpha \approx 33.1689^\circ \)
    \end{itemize}
\end{itemize}

\subsection{Secant Method Results}
\begin{itemize}
    \item \textbf{Using depth \( D_2 = 30 \) inches and initial guesses \( 10^\circ \) and \( 50^\circ \)}:
    \begin{itemize}
        \item Solution: \( \alpha \approx 33.1689^\circ \)
    \end{itemize}
\end{itemize}

In all three calculations, the results converge to approximately \( 33^\circ \).

% ============================================
\end{document}