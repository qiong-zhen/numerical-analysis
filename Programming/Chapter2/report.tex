\documentclass[a4paper]{article}
\usepackage{ctex} 
\usepackage[affil-it]{authblk}
\usepackage[backend=bibtex,style=numeric]{biblatex}
\usepackage{amsmath}
\usepackage{graphicx}


\usepackage{geometry}
\geometry{margin=1.5cm, vmargin={0pt,1cm}}
\setlength{\topmargin}{-1cm}
\setlength{\paperheight}{29.7cm}
\setlength{\textheight}{25.3cm}

\addbibresource{citation.bib}

\begin{document}
% =================================================
\title{Numerical Analysis Report1}

\author{谭希 3220100027
  \thanks{Electronic address: \texttt{3220100027@zju.edu.cn}}}
\affil{(数学与应用数学2201), Zhejiang University }


\date{Due time: \today}

\maketitle

% ============================================
\section{A}
\subsection{Interpolation.hpp}

ProblemFunction:Define a specific function:
\[
f(x) = \frac{1}{1 + ax^2}
\]
which is used for interpolation calculations.

Interpolation:
Manage the interpolation points \texttt{x\_points} and the predicted \texttt{y\_pre} values. The constructor generates a series of uniformly distributed \texttt{x} values.

NewtonInterpolation:Implements Newton's interpolation method. Allowing for interpolation based on given \texttt{x} and \texttt{y} values, or through the \texttt{ProblemFunction}.
HermiteInterpolator:Implements Hermite interpolation.

\section{B}
\begin{enumerate}
    \item An array \( N \) is defined, containing different sample sizes (2, 4, 6, 8).
    \item For each \( N[i] \), compute evenly distributed \( x \) values from -5 to 5.
    \item Use a \texttt{ProblemFunction} object to generate the corresponding function values.
    \item Then:
    \begin{itemize}
        \item Create a \texttt{NewtonInterpolation} object to execute Newton's interpolation method, setting the interpolation range and sample size.
        \item Call the \texttt{newtonInterpolation} method to perform the interpolation.
    \end{itemize}
\end{enumerate}


\section{C}

generates \texttt{n} Chebyshev nodes within the interval \([st, ed]\). 
The code implements interpolation of the function (defined by \texttt{ProblemFunction}) using Chebyshev nodes and Newton interpolation to obtain function values at these nodes

\section{D}

Create a \texttt{HermiteInterpolator} object by passing the known vectors \(x\), \(y\), and \(dy\), along with the interpolation range (from 0 to 13), and specify the number of interpolation points as 1000.

Call the \texttt{getPolynomial()} method to obtain the interpolation polynomial.  
Use the \texttt{getPointValue(10)} method to calculate and obtain the interpolated result at \(x = 10\).


Through a loop, calculate the slope (derivative value) between adjacent points and identify the maximum value among these slopes.

Determine whether the maximum derivative value exceeds 81 and output the corresponding information.


\section{E}

\section{F}

% ============================================
\end{document}