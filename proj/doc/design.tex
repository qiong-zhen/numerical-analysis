\documentclass[a4paper]{article}
\usepackage{ctex} 
\usepackage[affil-it]{authblk}
\usepackage{amsmath}
\usepackage{graphicx}
\usepackage{geometry}
\geometry{margin=1.5cm, vmargin={0pt,1cm}}
\setlength{\topmargin}{-1cm}
\setlength{\paperheight}{29.7cm}
\setlength{\textheight}{25.3cm}

\title{Numerical Analysis Report 1}
\author[1]{王丹}
\affil[1]{University, Email: \texttt{1901298538@tan.edu.cn}}
\date{\today}

\begin{document}
\maketitle

\section{程序结构说明}

\section{类及其结构}

\section{所用到的数学类理论}
\subsection{pp form splines}

从定理3.3可得令$m_i = s^\prime(f;x_i),for\ s\in \mathcal{S}_3^2$,对于$i=2,3,\cdots,N-1$,有:
\begin{equation}
    \lambda_i m_{i-1} +2m_i +\mu_i m_{i+1} = 3\mu_i f[x_i,x_{i+1}]+3\lambda_if[x_{i-1},x_i]
\end{equation}
式中
\begin{equation}
    \mu_i = \frac{x_i-x_{i-1}}{x_{i+1}-x_{i-1}}=\frac{x_i-x_{i-1}}{x_{i+1}-x_{i}+x_i-x_{i-1}} = \frac{h_i}{h_{i+1}+h_i}, \ \ \lambda_i = \frac{x_{i+1}-x_{i}}{x_{i+1}-x_{i-1}} = \frac{h_{i+1}}{h_{i+1}+h_i},
\end{equation}
\subsubsection{完全三次样条}
完全三次样条满足边界条件
\begin{equation}
    s^\prime(f;a) = f^\prime(a) \ \ 和 \ \ s^\prime(f;b)=b
\end{equation}
即给定曲线在区间两个端点的一阶导数。
令$b_i =3\mu_{i+1} f[x_{i+1},x_{i+2}]+3\lambda_{i+1}f[x_{i},x_{i+1}] ,i=1,\dots,N-2$则有:
\begin{equation}
\begin{bmatrix}2&\mu_2\\\lambda_3&2&\mu_3\\&&\ddots\\&&\lambda_i&2&\mu_i\\&&&&\ddots\\&&&&\lambda_{N-2}&2&\mu_{N-2}\\&&&&&\lambda_{N-1}&2\end{bmatrix}\begin{bmatrix}m_2\\m_3\\\vdots\\m_i\\\vdots\\m_{N-2}\\m_{N-1}\end{bmatrix}=\mathrm{b} = \begin{bmatrix}
    b_1-\lambda_2m_1 \\
    b_2 \\
    \vdots\\
    b_{i-1}\\
    \vdots \\
     b_{N-3}\\ 
     b_{N-2} - \mu_{N-1}m_N
\end{bmatrix}
\end{equation}
\subsubsection{自然三次样条}
自然三次样条已知端点两处的二阶导数值:
\begin{equation}
    s^{\prime \prime}(f;a)=0 \ \ s^{\prime \prime}(f;b)=0 
\end{equation}
由定理3.3知
\begin{equation}
    s_i(x) = f_i + (x-x_i)m_i + (x-x_i)^2 \frac{K_i-m_i}{x_{i+1}-x_i} + (x-x_i)^2 (x-x_{i+1})\frac{m_i + m_{i+1}-2K_i}{(x_{i+1}-x_i)^2}
\end{equation}
对上式求二阶导得
\begin{equation}
    s_i^{\prime\prime}(x) = 2\frac{K_i-m_i}{x_{i+1}-x_i} + 2(x-x_{i+1})\frac{m_i+m_{i+1}-2K_i}{(x_{i+1}-x_i)^2} + 4(x-x_{i})\frac{m_i+m_{i+1}-2K_i}{(x_{i+1}-x_i)^2} 
\end{equation}
整理得到:
\begin{equation}
    s_i^{\prime\prime}(x) = \frac{6x-4x_i-2x_{i+1}-2h_i}{h_i^2}m_i + \frac{6x-4x_i-2x_{i+1}}{h_i^2}m_{i+1}+K_i\frac{-12x+8x_i+4x_{i+1}+2h_i}{h_i^2}
\end{equation}
上式可化为:
\begin{equation}
    s_i^{\prime\prime}(x) = \frac{6x-2x_i-4x_{i+1}}{h_i^2}m_i+ \frac{6x-4x_i-2x_{i+1}}{h_i^2}m_{i+1} + \frac{-12x+6x_i+6x_{i+1}}{h_i^2}K_i
\end{equation}
若曲线在$[x_1,x_2]$之上,令$s_i^{\prime\prime}(x_1)=0$,于是有:
\begin{equation}
    s_1^{\prime\prime}(x_1) = -\frac{4}{h_1}m_0 - \frac{2}{h_1}m_1 + \frac{6}{h_1} \frac{f_2-f_1}{h_1}
\end{equation}
则有:
\begin{equation}
    2m_0+m_1 = 3f[x_1,x_2]-\frac{2}{h_1} s_1^{\prime\prime}(x_1) = 3f[x_1,x_2]
\end{equation}

同理我们可以得到$m_{N-1} + 2m_N = 3f[x_{N-1},x_N]$

于是有:
\begin{equation}
    \left\{ 
    \begin{array}{c}
    2m_1+m_2 = 3f[x_1,x_2] =g_0      \\
          m_{N-1} + 2m_N = 3f[x_{N-1},x_N]=g_{N-1}
    \end{array}
    \right.
\end{equation}

因此得到:


\begin{equation}
\begin{bmatrix}2&1
\\\lambda_2&2&\mu_2
\\&&\ddots\\&&\lambda_i&2&\mu_i\\&&&&\ddots\\&&&&\lambda_{N-1}&2&\mu_{N-1}\\&&&&&1&2\end{bmatrix}\begin{bmatrix}m_1\\ m_2\\m_3\\\vdots\\m_i\\\vdots\\m_{N-1}\\mN\end{bmatrix}=\mathrm{b} = \begin{bmatrix}
    g_0 \\
    b_1 \\
    \vdots \\
    b_{i -1}\\
    \vdots \\
    b_{N-2}\\
    g_{N-1}
\end{bmatrix}
\end{equation}
其中

\subsubsection{周期三次样条}
周期三次样条满足:
\begin{equation}
    s(f;b) = s(f;a),s^\prime (f;b) = s^\prime (f;a),s^{\prime\prime} (f;b) = s^{\prime\prime} (f;a)
\end{equation}
即
\begin{equation}
    s(x_1) = s(x_N) \ \ s^\prime(x_1) = s^\prime(x_N) \ \ s^{\prime\prime}(x_1) = s^{\prime\prime}(x_N)
\end{equation}

由一阶导相等得条件可得:
\begin{gather}
    s_1^\prime(x_1) = m_1 \\
    s_{N-1}^\prime(x_N) = m_N 
\end{gather}

由二阶导相等得条件可得:
\begin{equation}
    -\frac{4}{h_1}m_1 - \frac{2}{h_1}m_2 + \frac{6}{h_1}K_1 = \frac{2}{h_{N-1}}m_{N-1}+\frac{4}{h_{N-1}}m_N -\frac{6}{h_{N-1}^2}K_{N-1}
\end{equation}
进一步推出:
\begin{equation}
  \frac{1}{h_1} m_2 + \frac{1}{h_{N-1}} m_{N-1} +(\frac{1}{h_{N-1}}+ \frac{1}{h_1})m_N = 3\frac{1}{h_1}K_1 +3\frac{1}{h_{N-1}}K_{N-1}
\end{equation}
令$\lambda_N =\frac{x_2-x_1}{x_2-x_1 +x_N-x_{N-1}}$和$\mu_N = \frac{x_N-x_{N-1}}{x_2-x_1 +x_N-x_{N-1}}$于是有:
\begin{gather}
    \left\{
    \begin{array}{c}
    m_1 = m_N \\
        \mu_N m_2 + 2\lambda_N m_{N-1} + 2m_N = 3(\mu_Nf[x_1,x_2]+\lambda_N f[x_{N-1},x_N])
    \end{array}
    \right.
\end{gather}
于是有方程组:
\begin{equation}
\begin{bmatrix}2&\mu_2\\\lambda_3&2&\mu_3\\&&\ddots\\&&\lambda_i&2&\mu_i\\&&&&\ddots\\&&&&\lambda_{N-2}&2&\mu_{N-2}\\&&&&&\lambda_{N-1}&2 &\mu_{N-1} \\
\mu_N&&&&&&2\lambda_N &2 \end{bmatrix}
\begin{bmatrix}m_2\\m_3\\\vdots\\m_i\\\vdots\\m_{N-2}\\m_{N-1}\\
m_N\end{bmatrix}=\mathrm{b} = \begin{bmatrix}
    b_1 \\
    b_2 \\
    \vdots \\
    b_i \\
    \vdots \\
    b_{N-2}\\
    g_N
\end{bmatrix}
\end{equation}
\subsection{B-splines}
由定理3.57可知,存在唯一的$S(x)\in \mathcal{S}_3^2$对$f(x)$在点$1,2,\cdots,N$进行插值,并有$S^\prime(1)=f^\prime(1)$和$S^\prime(N) = f^\prime(N)$,因此B-样条插值函数为:
\begin{equation}
    S(x) = \sum_{i=-1}^{N}a_i B_{i}^3(x)
\end{equation}
系数$a_i$通过求解线性方程组$Ma=b$获得。
\begin{equation}
    f(t_i)=S(t_i) = a_{i-2}B^3_{i-2}(t_i)+a_{i-1}B^3_{i-1}(t_i)+a_{i}B^3_{i}(t_i)
\end{equation}
在上述等式的基础之上引入边界条件就可求解系数$a= [a_{-1},a_0,a_1,\cdots,a_{N}]^T$
\subsubsection{完全三次样条}
对于完全三次样条来讲,存在边界条件为在端点处的一阶导数相等:
\begin{equation}
    S^\prime(t_1)= f^\prime(t_1)
\end{equation}
\begin{equation}
    S^\prime(t_N) = S^\prime(t_N)
\end{equation}
且$S^\prime(x) = \sum_{i=-1}^{N}a_i B_i^{\prime 3}(x)$
于是线性方程组为:
\begin{equation}
    M = \begin{bmatrix}
        B_{-1}^{\prime 3}(t_1)& B_{0}^{\prime 3}(t_1) &B_{1}^{\prime 3}(t_1)\\
        B_{-1}^3(t_1) &B_{0}^3(t_1)&B_{1}^3(t_1)\\
        &\ddots &\ddots&\ddots&\\
        &&B_{N-2}^3(t_N)&B_{N-1}^3(t_N)&B_{N}^3(t_N)\\
        &&B_{N-2}^{\prime 3}(t_N)&B_{N-1}^{\prime3}(t_N)&B_{N}^{\prime3}(t_N)\\
    \end{bmatrix}
\end{equation}
\begin{equation}
    b^T = [f^\prime(t_1),f(t_1),\cdots,f(t_N),f^{\prime}(t_N)]
\end{equation}

\subsubsection{周期边界条件}
对于周期边界条件而言,有:
\begin{equation}
    S(t_1) = S(t_N)
\end{equation}

\begin{equation}
    S^\prime(t_1) = S^\prime(t_N)
\end{equation}

\begin{equation}
    S^{\prime \prime}(t_1) = S^{\prime \prime}(t_N)
\end{equation}
于是有:
\begin{equation}
    M = \begin{bmatrix}
        -B_{-1}^{\prime3}(t_1)&-B_{0}^{\prime3}(t_1)&-B_{1}^{\prime3}(t_1) &0&\cdots&0&B_{N-2}^{\prime3}(t_N)&B_{N-1}^{\prime3}(t_N)&B_{N}^{\prime3}(t_N) \\
        B_{-1}^3(t_1)&B_{0}^3(t_1) &B_{1}^3(t1) \\
        & \ddots & \ddots & \ddots \\
        &&&&&0 & B_{N-2}^3(t_N)&B_{N-1}^3(t_N)&B_{N}^3(t_N)\\
        -B_{-1}^{\prime\prime3}(t_1)&-B_{0}^{\prime\prime3}(t_1)&-B_{1}^{\prime\prime3}(t_1) &0&\cdots&0&B_{N-2}^{\prime\prime3}(t_N)&B_{N-1}^{\prime\prime3}(t_N)&B_{N}^{\prime\prime3}(t_N)
    \end{bmatrix}
\end{equation}
\begin{equation}
    b^T = [0,f(t_1),\cdots,f(t_{N-1}),f(t_N),0]^T
\end{equation}

\subsubsection{自然样条}
自然边界条件为:
\begin{equation}
    S^{\prime \prime}(t_1) = S^{\prime\prime}(t_N)=0
\end{equation}
于是通过求解线性方程组$Ma=b$来得到B-slpines的系数:
\begin{equation}
    M=\begin{bmatrix}
        B_{-1}^{\prime \prime 3}(t_1)& B_{0}^{\prime \prime 3}(t_1) &B_{1}^{\prime \prime 3}(t_1)\\
        B_{-1}^3(t_1) &B_{0}^3(t_1)&B_{1}^3(t_1)\\
        &\ddots &\ddots&\ddots&\\
        &&B_{N-2}^3(t_N)&B_{N-1}^3(t_N)&B_{N}^3(t_N)\\
        &&B_{N-2}^{\prime \prime 3}(t_N)&B_{N-1}^{\prime \prime 3}(t_N)&B_{N}^{\prime \prime 3}(t_N)\\
    \end{bmatrix}
\end{equation}
\begin{equation}
    b^T = [0,f(t_1),\cdots,f(t_N),0]^T
\end{equation}
\end{document}
